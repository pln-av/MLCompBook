% short cut for equation environment
\newcommand{\eq}[1]{\begin{equation}#1\end{equation}} 
\newcommand{\eqa}[1]{\begin{align*}#1\end{align*}}
% -----------------------------------------------------
% Math font macros
% -----------------------------------------------------
\newcommand{\mb}[1]{ \mathbf{#1} }
\newcommand{\mc}[1]{ \mathcal{#1} }
\newcommand{\mbs}[1]{ \boldsymbol{#1} }

% -----------------------------------------------------
% Math operators
% -----------------------------------------------------
\newcommand{\dwrt}[2]{\frac{d#1}{d#2}}
\newcommand{\pwrt}[2]{\frac{\partial#1}{\partial#2}}
\newcommand{\func}[1]{\left(#1\right)}
\newcommand{\ith}[1]{#1^{\scriptscriptstyle \textrm{th}}}
\newcommand{\order}[1]{\mathcal{O}\func{#1}}
\newcommand{\expected}[0]{\mathop{\mathbb{E}}}
\newcommand{\covar}[0]{\mathop{\mathbb{Cov}}}
\newcommand{\variance}[0]{\mathop{\mathbb{Var}}}

% -----------------------------------------------------
% Linear Algebra macros
% -----------------------------------------------------

\renewcommand*{\arraystretch}{0.9}
\newcommand{\msize}[2]{\left(#1 \times #2\right)}
\newcommand{\transpose}[1]{\ensuremath{#1^{\scriptscriptstyle T}}}
\newcommand{\inverse}[1]{\ensuremath{#1^{\scriptscriptstyle -1}}}
\newcommand{\determinant}[1]{\mathtt{det}\func{#1}}
% mat for general matrix environment
% matt where 't' is for 'three'
% matx which uses [h|v]dots to show size
\newcommand{\mat}[1]{
    \begingroup % keep the change local
    \setlength\arraycolsep{1.5pt}
    \begin{pmatrix}#1\end{pmatrix}
    \endgroup
} 

% columns
\newcommand{\colt}[1]{\mat{#1_1 \\ #1_2 \\ #1_3}}
\newcommand{\colx}[2]{\mat{#1_1 \\ #1_2 \\ \vdots \\ #1_#2}}
\newcommand{\col}[1]{ \mathtt{c}\func{#1} }
\newcommand{\matcol}[3] 
{ 
    \mat{ #1_{1#2} \\ #1_{2#2} \\ \vdots \\ #1_{#3#2} }
}
% rows 
\newcommand{\rowt}[1]{\mat{#1_1 & #1_2 & #1_3}}
\newcommand{\rowx}[2]{\mat{#1_1 & #1_2 & \hdots & #1_#2}}
\newcommand{\row}[1]{ \mathtt{r}\left(#1\right) }
\newcommand{\matrow}[3] 
{ 
    \mat{ #1_{#21} & #1_{#22} & \hdots & #1_{#2#3} }
}
% matrices
\newcommand{\matt}[1]
{\mat{
    #1_{11} & #1_{12} & #1_{13} \\
    #1_{21} & #1_{22} & #1_{23} \\
    #1_{31} & #1_{32} & #1_{33}
    } 
}
\newcommand{\matx}[3]
{\mat{
    #1_{11} & #1_{12} & \cdots & #1_{1 #3} \\
    #1_{21} & #1_{22} & \cdots & #1_{2 #3} \\
    \vdots & \vdots & \ddots & \vdots \\
    #1_{#21} & #1_{#2 2} & \cdots & #1_{#2 #3}
    } 
}
